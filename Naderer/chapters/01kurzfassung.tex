\chapter{Kurzfassung}
\label{cha:Kurzfassung}
Cloud Computing bietet Software Herstellern bisher nicht vorhandene technische Möglichkeiten zur Erstellung von hoch skalierbaren Hardware Infrastrukturen. Um diese Möglichkeiten mit der bestehenden Software \textit{Microsoft Dynamics 365 Business Central} zu nutzen, und gleichzeitig die Vorteile des Systems zu wahren, wurde ein neues Programmierkonzept eingeführt. Code-Änderungen, wie sie bisher die Norm waren, sind in einem Cloud-System nicht akzeptabel. Daher wurde das Konzept der erweiterungsbasierten Programmierung mit ExtensionsV2 eingeführt. ExtensionsV2 erlaubt es Entwicklern, die Grundfunktionalität des ERP-Systems zu erweitern, ohne die Stabilität und Updatefähigkeit des Systems negativ zu beeinflussen.


Diese Arbeit vermittelt einen Überblick über das Programmsystem \textit{Microsoft Dynamics 365 Business Central}, der verwendeten Schichtenarchitektur und vergleicht die Programmierkonzepte hinter den beiden Sprachen C/AL und AL.


Nach der Definition des Begriffs \textit{ERP-System} und der geschichtlichen Entwicklung des Systems \textit{Microsoft Dynamics 365 Business Central} werden die verschiedenen Arten von Objekten, die als Grundbausteine des Systems fungieren beleuchtet. Im Hauptteil wird das neue Programmiermodell mit ExtensionsV2 mit der Code-Anpassung unter C/AL verglichen.  Um einen quantifizierbaren Vergleich zu ermöglichen wird eine Erweiterung für das System mit beiden Arten der Programmierung umgesetzt. Da beide Implementierungen schlussendlich dieselbe Funktionalität zur Verfügung stellen, kann so speziell auf die Unterschiede zwischen den beiden Programmierparadigmen eingegangen werden. Anschließend werden Laufzeitmessungen durchgeführt, um Differenzen zwischen den Implementierungen aufzudecken. Abschließend werden die Implikationen bezüglich zeitgemäßer Entwicklungswerkzeuge wie Source Code Management und CI/CD diskutiert, die mit dem dateibasierten Entwicklungsprozess in der erweiterungsbasierten Programmierung in Visual Studio Code und AL einher gehen.