\chapter{Einleitung}
\label{cha:Einleitung}

\section{Motivation}
\label{sec:Motivation}

ERP-Systeme sind heute aus dem wirtschaftlichen Umfeld nicht mehr wegzudenken. Bereits in den 1970er Jahren erkannten Wirtschaftstreibende Potential darin, ihre Prozesse und Unternehmensdaten zu digitalisieren. Während in den 1970er und 1980er Jahren innerhalb der einzelnen Abteilungen eines Unternehmens verschiedene Software-Lösungen zum Einsatz kamen, ist man sich mittlerweile einig, dass eine zentrale Applikation zur Verwaltung aller unternehmerisch wichtigen Daten und Prozessschritte große Vorteile liefert. Diese Erkenntnis hatte zur Folge, dass kleine Softwarehersteller immer mehr ins Wanken gerieten, da von der Wirtschaft allumfassende Software-Giganten gefordert wurden, die eine Vielzahl von Anforderungen aus den verschiedensten Anwendungsdomänen zu erfüllen haben. Solche Systeme können mit den meist begrenzten Ressourcen und Branchenwissen kleinerer Hersteller nicht entwickelt werden. Gleichzeitig entstanden durch die gewachsenen Anforderungen umfassende Softwaresysteme einiger größerer Hersteller, hier sind vor allem Marktführer SAP, aber auch Microsoft mit seiner Dynamics Sparte zu nennen.

Wer heute in einem Unternehmen mit der Einführung eines ERP-Systems betraut wird, muss sich intensiv mit den verschiedenen erhältlichen Lösungen auseinander setzen. Denn neben Lizenzierung und finanziellen Aspekten, ist zu erarbeiten, welche Systeme die bestehenden Prozesse des Unternehmens am Besten abbilden. Systeme bilden Geschäftsprozess meist auf eine bestimmte Art ab. Sollte diese nicht mit dem Vorgehen des Unternehmens überein stimmen, bleiben meist nur zwei Auswegen offen. Entweder das Unternehmen passt seine Prozesse an die Vorgabe des Systems an, oder das System muss entsprechend angepasst werden, um den Ansprüchen des Unternehmensprozesses zu genügen.

Und genau hier spielt die Anpassbarkeit und Erweiterbarkeit eines Systems die zentrale Rolle. Gerade branchenspezifische und insbesondere unternehmensspezifische Prozesse müssen meist erst programmiert und in das System integriert werden. Programmierarbeiten und Änderungen am Standardsystem sind meist Aufwendig, und stellen so ein nicht zu vernachlässigendes finanzielles Risiko dar.

Um die Aspekte der Erweiterbarkeit und Anpassungsmöglichkeit möglichst gut zu erfüllen, entschied sich Microsoft im ERP-System Microsoft Dynamics NAV bereits in sehr frühen Versionen, zertifizierten Entwicklern freien Zugang zum Applikationscode zu gewähren. So können Entwickler die gesamte Geschäftslogik des Systems je nach Unternehmensanforderungen abändern und erweitern, in dem Sie neuen Programmcode hinzufügen, oder den Standardcode von Microsoft anpassen oder löschen. Dies hat zum Einen zur Folge, dass Entwickler mächtige Applikationen erstellen können, und sich diese direkt in das bestehende System integrieren lassen. Allerdings kommen mit diesen umfassenden Möglichkeiten auch Probleme auf. Je weiter die oft über Jahrzehnte verwendeten und erweiterten Systeme von der Codebasis von Microsoft abweichen, desto aufwendiger, fehleranfälliger und teurer ist es, diese Systeme mit den Aktualisierungen des Herstellers zu versorgen, die periodisch in das System integriert werden müssen.  

Um die Systeme update-fähig zu halten, und gleichzeitig ein Entwicklungsmodell zu schaffen, dass auch in der Cloud-Variante des ERP-Systems funktionieren kann, wurde mit Dynamics NAV 2017 erstmals das Konzept der Erweiterungsprogrammierung mit ExtensionsV1 für Dynamics NAV vorgestellt. ExtensionsV1 ist ein gänzlich neuer Ansatz für das ERP-System zu programmieren, und hat konzeptionell viele Vorteile gegenüber der konventionellen prozeduralen Entwicklung, ist aber mittlerweile aufgrund einiger technischer Schwierigkeiten obsolet.

Mit der in 2018 veröffentlichten Version - ExtensionsV2 - sind nicht nur viele der technischen Mängel behoben, ExtensionsV2 kommt auch mit einer neuen Programmiersprache und Entwicklungsumgebung.

\section{Zielsetzung}
\label{sec:Zielsetzung}
Im Rahmen dieser Arbeit, soll ein Überblick über die Plattform Dynamics NAV bzw. die Cloud-Variante Dynamics 365 Business Central und Programmierung dieser Systeme gegeben werden. Hierfür wird erst eine Übersicht über das Gesamtsystem und seine Schichtenarchitektur vermittelt. Anschließend wird das Konzept der Erweiterungsentwicklung mit ExtensionsV2 mit der konventionellen prozeduralen Entwicklung verglichen. Dies passiert anhand eines Beispiels, dass auf beide Arten gelöst wird. Einerseits wird die Anwendung in der Entwicklungsumgebung C/SIDE mit C/AL entwickelt. Andererseits wird anhand der Aufgabenstellung eine Erweiterung mit ExtensionV2 in VisualStudio Code und der neuen Sprache AL erstellt. Hierbei liegt der Fokus nicht darauf, kleine syntaktischen Unterschiede zwischen den Sprachen hervorzuheben, sondern Neuerungen in der neuen Sprache AL zu beleuchten, und konzeptionelle Unterschiede zwischen den beiden Programmierparadigmen aufzuzeigen und zu bewerten.

Dies erfolgt anhand von statischen und dynamischen Code-Metriken sowie Performanzmessungen. Zusätzlich wird auch diskutiert, welche Vor- und Nachteile sich durch die nun neue Datei-basierte Codeverwaltung hinsichtlich der Einbindung und Nutzung von SCM und CI/CD Systemen ergeben. In einem letzten Block wird danach das neue Event-basierte Programmiermodell der Erweiterungsentwicklung mit ExtensionsV2 diskutiert, und die Vor- und Nachteile beleuchtet, die mit dem Wechsel von Code-Anpassung hin zu Code-Erweiterung einher gehen.

