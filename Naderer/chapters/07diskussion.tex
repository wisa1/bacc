\chapter{Zusammenfassung \& Ausblick}
\label{cha:Diskussion}

\section{Zusammenfassung}
Im Laufe dieser Arbeit wurde eine voll funktionale Erweiterung zur Verwaltung von Treuepunkten für Stammkunden entwickelt. Dafür wurden vorerst die verschiedenen Arten von Applikationsobjekten beleuchtet. Im Weiteren Verlauf der Arbeit wurde die Erweiterung wurde sowohl durch Codeanpassung in C/AL, als auch erweiterungsbasiert mit der neuen Sprache AL umgesetzt. Aus den Tests geht hervor, dass beide Varianten wie erwartet dasselbe Verhalten aufweisen.  Auch die Laufzeitmessungen der beiden Varianten sind bis auf die im Kapitel Evaluierung \& Diskussion hervorgehobenen architekturbedingten Sonderfälle durchaus auf dem selben Niveau. Das größte Problem mit AL ist die Abhängigkeit auf von Microsoft veröffentlichten Ereignissen. Nichts desto trotz überwiegen die vielen Vorteile, angefangen von automatisierbaren Updates und einer zeitgerechten Entwicklungsumgebung, bis hin zur Möglichkeit bisher langwierige Aufgaben wie Erstellung von online fähigen Testsystemen zu automatisieren.

\section{Ausblick}
Zukunftsweisend - C/AL wird eingestellt
Alte Entwickler, träge in bezug auf Techn. Wechsel
Mit neuem Verkarktungsmodell chancenreich f. Entwickler

