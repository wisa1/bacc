\chapter{Zusammenfassung \& Ausblick}
\label{cha:Diskussion}

\section{Zusammenfassung}
Im Laufe dieser Arbeit wurde eine voll funktionale Erweiterung zur Verwaltung von Treuepunkten für Stammkunden entwickelt. Dafür wurden vorerst die verschiedenen Arten von Applikationsobjekten beleuchtet. Im weiteren Verlauf der Arbeit wurde die Erweiterung sowohl durch Codeanpassung in C/AL, als auch erweiterungsbasiert mit der neuen Sprache AL umgesetzt. Aus den Tests geht hervor, dass beide Varianten wie erwartet das gleiche Verhalten aufweisen. Auch die Laufzeitmessungen der beiden Varianten sind bis auf die im Kapitel Evaluierung \& Diskussion hervorgehobenen architekturbedingten Sonderfälle durchaus auf dem selben Niveau. Mit dem erweiterten Funktionsumfang von AL im Vergleich zu C/AL lassen sich nun viele Probleme, die unter C/AL nur mit externen Klassenbibliotheken lösbar waren, nun direkt mithilfe der neuen Sprachkonstrukte lösen. Das größte Problem mit AL ist derzeit noch die Abhängigkeit auf von Microsoft veröffentlichten Ereignissen. Nichts desto trotz überwiegen die vielen Vorteile, angefangen von automatisierbaren Updates und einer zeitgerechten Entwicklungsumgebung, bis hin zur Möglichkeit bisher langwierige Aufgaben wie die Erstellung von online fähigen Testsystemen zu automatisieren.

\section{Ausblick}
 Die in dieser Arbeit verwendeten Hauptversion 14 des Systems ist die letzte, in der Code Anpassungen in C/AL noch möglich sind. Mit der kommenden Hauptversion im Oktober 2019 werden Anpassungen des Systems nur noch via in AL entwickelten Erweiterung unterstützt \cite{stefanetti2019}. Um künftige Iterationen des Systems zu unterstützen, müssen Entwickler somit ihre C/AL Anpassungen in AL Erweiterungen übersetzen. Traditionell sind die Partnerunternehmen bei der Umsetzung grober technologischer Änderungen jedoch sehr träge. Selbst die vergleichsweise kleine Umstellung auf die 3-Schichten-Architektur im Jahr 2009 beschäftigte Entwickler oft mehrere Jahre. Dementsprechend ist auch bei dieser nun größeren Umstellung, inklusive dem Sprung in die Cloud damit zu rechnen, dass der Umstieg einige Jahre in Anspruch nehmen wird. Denn Partnerunternehmen müssen natürlich weiterhin ihre Bestandskunden mit älteren Versionen unterstützen, und sollen nebenher Neuprojekte auf Cloudbasis mit einer neuen Programmiersprache und einem stark abgewandelten Entwicklungskonzept umsetzen. Dennoch ist davon auszugehen, dass sich die Cloudvariante von Microsoft Dynamics 365 Business Central zumindest mittelfristig durchsetzen wird, und es somit für Entwickler an der Zeit ist, sich der vertrauten C/AL Welt ein Stück weit abzuwenden, und den Sprung in das  erweiterungsbasierte Konzept von AL zu wagen.

