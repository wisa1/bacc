\chapter*{Abstract}

Cloud computing offers software manufacturers previously unavailable technical possibilities for creating highly scalable hardware infrastructures. To take advantage of these capabilities with the existing software \textit{Microsoft Dynamics 365 Business Central} while maintaining the benefits of the system, a new programming concept has been introduced. Code changes, as previously the norm, are unacceptable in a cloud system. Therefore, the concept of extension-based programming with ExtensionsV2 has been introduced. ExtensionsV2 allows developers to extend the basic functionality of the ERP system without negatively affecting the stability and updating capability of the system.


This thesis provides an overview of the program system \textit {Microsoft Dynamics 365 Business Central}, the used layer architecture and compare s the programming concepts behind the two languages C/AL and AL).


Following the definition of the term \textit{ERP system} and the historical development of the system \textit{Microsoft Dynamics 365 Business Central}, the various types of objects that function as the basic building blocks of the system are examined. The main part of this thesis compares the new extension based programming model using the ExtensionsV2 system to the code adaptation with C/AL. To allow a quantifiable comparison, an extension for the system is implemented with both types of programming. Since both implementations ultimately provide the same functionality, it is possible to specifically address the differences between the two programming paradigms. Next, runtime measurements are performed to reveal differences between the implementations. Finally, the implications of contemporary development tools such as source code management and CI/CD are discussed, which go hand in hand with the new file-based development process in extension-based programming in Visual Studio Code and AL.