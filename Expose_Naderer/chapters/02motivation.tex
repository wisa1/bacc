\chapter{Motivation}
\label{cha:Motivation}

ERP-Systeme sind heute aus dem wirtschaftlichen Umfeld nicht mehr wegzudenken. Wo in den 1970er und 1980er Jahren innerhalb der einzelnen Abteilungen von Unternehmen verschiedene Software-Lösungen verwendet wurden, ist man sich mittlerweile einig, dass eine zentrale Applikation zur Verwaltung aller unternehmerisch wichtigen Daten große Vorteile liefert.
Gleichzeitig hatte dieses Umdenken zur Folge, dass kleine Softwarehersteller ins wanken gerieten - da seitens der Wirtschaft allumfassende Software-Giganten gefordert wurden, die mit eingeschränkten Ressourcen und Branchenwissen nicht entwickelt werden konnten. Genau diese Großsysteme werden neben dem Platzhirsch SAP und einigen anderen Unternehmen auch von Microsoft mit seiner Dynamics Sparte geliefert.

Vor der Kaufentscheidung für ein ERP-System gilt es neben den finanziellen Aspekten auch herauszufinden, welche Systeme die bestehenden Prozesse es Unternehmens am Besten abbilden. Und genau hier spielt die Anpassbarkeit und Erweiterbarkeit eines Systems eine zentrale Rolle. Systeme bilden oft Geschäftsprozesse auf ihre eigene Art ab, und zwingen so den Anwender bestehende Prozesse anzupassen, oder das System zu ändern. Gerade bei Branchenspezifischen Abläufen muss hier meist die Logik des ERP-Systems geändert oder erweitert werden. Solche Änderungen sind in den meisten Fällen jedoch teuer und zeitintensiv. Da alle großen Systeme darüber hinaus auch periodisch mit Updates versorgt werden, um gegenüber rechtlichen Änderungen flexibel zu bleiben, können sich Eingriffe in die Logik auch negativ auf die Upgradefähigkeit des Systems auswirken. 



